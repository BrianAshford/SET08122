\documentclass[10pt, a4paper, twosize]{article}
%\documentclass[12pt, a4paper, twoside]{book}

\usepackage{helvet}
\usepackage{hyperref}
\usepackage{graphicx}
\usepackage{listings}
\usepackage{textcomp}
\usepackage[
	a4paper,
	outer=2cm,
	inner=4cm,
	top=2cm,
	bottom=2cm
]{geometry}
\usepackage{float}
\usepackage{tabularx}
\usepackage[disable]{todonotes}
\usepackage{color, soul}
\usepackage{amsmath}
\usepackage{algorithmicx}
\usepackage[noend]{algpseudocode}
\usepackage{algorithm}
\usepackage{framed}
\usepackage{subcaption}
\usepackage{titlepic}
\usepackage{fancyhdr}
\usepackage[simplified]{styles/pgf-umlcd}
\usepackage{shorttoc}
\usepackage{url}
\usepackage{paralist}

\definecolor{grey}{rgb}{0.9, 0.9, 0.9}
\definecolor{dkgreen}{rgb}{0,0.6,0}
\definecolor{dkred}{rgb}{0.6,0,0.0}

\lstdefinestyle{DOS}
{
    backgroundcolor=\color{black},
    basicstyle=\scriptsize\color{white}\ttfamily,
    stringstyle=\color{white},
    keywords={}
}

\lstdefinestyle{makefile}
{
    numberblanklines=false,
    language=make,
    tabsize=4,
    keywordstyle=\color{red},
    identifierstyle= %plain identifiers for make
}

\lstset{
  language=Java,                % the language of the code
  basicstyle=\footnotesize\ttfamily,
  numbers=left,                   % where to put the line-numbers
  stepnumber=1,                   % the step between two line-numbers. If it's 1, each line
  numbersep=5pt,                  % how far the line-numbers are from the code
  backgroundcolor=\color{white},      % choose the background color. You must add \usepackage{color}
  showspaces=false,               % show spaces adding particular underscores
  showstringspaces=false,         % underline spaces within strings
  showtabs=false,                 % show tabs within strings adding particular underscores
  frame=single,                   % adds a frame around the code
  rulecolor=\color{black},        % if not set, the frame-color may be changed on line-breaks within not-black text (e.g. comments (green here))
  tabsize=2,                      % sets default tabsize to 2 spaces
  captionpos=b,                   % sets the caption-position to bottom
  breaklines=true,                % sets automatic line breaking
  breakatwhitespace=false,        % sets if automatic breaks should only happen at whitespace
  keywordstyle=\color{blue},          % keyword style
  commentstyle=\color{dkgreen},       % comment style
  stringstyle=\color{dkred},         % string literal style
  columns=fixed,
  extendedchars=true,
  frame=single,
}

%\renewcommand{\chaptername}{Topic}

% New definitions
\algnewcommand\algorithmicswitch{\textbf{switch}}
\algnewcommand\algorithmiccase{\textbf{case}}
\algnewcommand\algorithmicassert{\texttt{assert}}
\algnewcommand\Assert[1]{\State \algorithmicassert(#1)}%
% New "environments"
\algdef{SE}[SWITCH]{Switch}{EndSwitch}[1]{\algorithmicswitch\ #1\ \algorithmicdo}{\algorithmicend\ \algorithmicswitch}%
\algdef{SE}[CASE]{Case}{EndCase}[1]{\algorithmiccase\ #1}{\algorithmicend\ \algorithmiccase}%
\algtext*{EndSwitch}%
\algtext*{EndCase}%

\pagestyle{fancy}
\fancyhf{}
\fancyhead[RO, LE]{\small \rightmark}
\fancyfoot[RO, LE]{\small \thepage}

\begin{document}

%\frontmatter

\begin{titlepage}
\vspace*{5cm}
\begin{center}
\includegraphics[width=.5\textwidth]{images/EdNapUniLogoCMYK}~\\[1cm]

\textsc{\Large Edinburgh Napier University}\\[1.5cm]

\textsc{\LARGE \bfseries SET09117 Algorithms \& Data Structures}\\[0.5cm]

\hrulefill \\[0.4cm]
{\huge \bfseries Lab 1 - Introduction \& Learning Environment \\[0.4cm] }
\hrulefill \\[1.5cm]

\begin{minipage}{0.4\textwidth}
\begin{flushleft} \large
\textbf{Dr Simon Wells} \\
\end{flushleft}
\end{minipage}

\vfill

\end{center}
\end{titlepage}

%\shorttoc{Overview}{0}

%\setcounter{tocdepth}{2}
%\cleardoublepage
%\tableofcontents
%\listoffigures
%\listofalgorithms
%\addtocontents{toc}{~\hfill\textbf{Page}\par}

%\mainmatter

%\input{sections/labs/04_ui}

\section{Aims}
\paragraph{} At the end of the practical portion of this topic you will:

\begin{itemize}
\item Be able to log into the University system
\item Be able to run simple programs from the command line
\item Remember how to program simple software
\item Have retrieved some supplementary texts from an online publisher
\end{itemize}


\begin{framed}
{\bf{NOTICE:} This is an introductory lab that is designed to ease ourselves back into programming and make sure that we are comfortable with our learning environment before we start learning new things in earnest. This is partly because the first lab this year is before the first lecture, but also to give a supportive introduction to the module for our direct entrants.}  
\end{framed}


\section{Activities}

\subsection{Running software from the Command line}

\paragraph{} Much of the example code we will be working with is short and simple and doesn't require a complex development environment which can, sometimes, make simple ideas seem much more complicated than necessary. So our default learning environment will be a programming language, both Python and Java are available on Universsty machines, of your choice. If you are using your own laptop, then you probably have some languages installed that you can use. You should choose a language that you are familiar with, however, this is also an opportunity to learn a new language, so have a think about which to do.

\subsubsection{Python}
\paragraph{} Python is available in the Napier Labs. Select the following icon

\includegraphics{images/software_icon}

\paragraph{} In the window that opens, scroll through until you find Python. It should look something similar to this:

\includegraphics[width=.8\textwidth]{images/sofware_window}

\paragraph{} Double click the Python icon and a Python shell should open. This shell is a Read-Evaluate-Print-Loop, known as a REPL (usually pronounced \emph{re-pul}). You can type in Python code and it will be processed and the result printed to the screen. The shell should look like this:

\includegraphics[width=.8\textwidth]{images/python_repl}

\paragraph{} If you've not used Python before but want to give it a go then I'd suggest quickly skipping to the background reading section, downloading ``Data Structures and Algorithms with Python'' then returning here and working through the first chapter which gives a quick introduction to Python. 

\paragraph{} Alternatively, you could learn some Python by exploring some of the excellent online `learn Python' tutorials that are available. Some of them are interactive, so you can type directly into the web-site and get results. Others just give you exercises to do yourself and you can do those exercises in Python on our Levinux platform. Here are a few, in order of usefulness according to me, but many more are just a Google search away:

\begin{enumerate}
\item Learn Python the hard way: \url{http://learnpythonthehardway.org/book/}
\item The Python Practice Book: \url{http://anandology.com/python-practice-book/index.html}
\item A Byte of Python: \url{http://www.swaroopch.com/notes/python/}
\item Code Combat: \url{https://codecombat.com/}
\item Python for you and me: \url{http://pymbook.readthedocs.org/en/latest/}
\item The Hitchhiker's Guide to Python:\\ \url{http://docs.python-guide.org/en/latest/#the-hitchhiker-s-guide-to-python}
\item Hands-On Python Tutorial: \url{http://anh.cs.luc.edu/python/hands-on/3.1/handsonHtml/index.html}
\item The Python Challenge: \url{http://www.pythonchallenge.com/} - Some quite touch challenges that you can solve using Python (NB. Assumes you already know what you are doing)
\item Python Tutor: \url{http://www.pythontutor.com/} - Helps visualise the execution of Python code. Again, assumes that you have some prior knowledge of Python syntax.
\end{enumerate}

\paragraph{} The links towards the top of the list are aimed at those completely new to the Python language whereas those further down the list will help those who have some Python knowledge already.

\subsubsection{Java}

\paragraph{} The Java JDK is already installed on lab machines. Open a new command line by navigating to the start menu and typing in cmd.exe then pressing return.


\includegraphics[width=.8\textwidth]{images/start_cmd}

\paragraph{} Your new terminal should look like this:

\includegraphics[width=.8\textwidth]{images/cmd}

\paragraph{} If you type javac and press return and get the following output:

\begin{lstlisting}[style=DOS]
    C:\>javac
    'javac' is not recognized as an internal or external command,
    operable program or batch file.
    C:\>
\end{lstlisting}

\paragraph{} then you need to set your path. Type the following into the command line then press return:

\begin{lstlisting}[style=DOS]
    C:\> set "path=%path%;c:\program files\Java\jdk1.8.0_121\bin"
\end{lstlisting}


\paragraph{} This will have set the path to the JDK for this session. Do some google research to discover how to set this permanently using an environment variable, but for now this will suffice. However you may need to reset your path each time you open a new command line.

\paragraph{} You can now run a simple Java program by typing some Java source code into a file, saving it then compiling and running it from the command line, e.g.

\begin{lstlisting}
public class HelloWorld {

	public static void main(String[] args) {
		System.out.println("Hello Algorithms & Data Structures");
	}
}
\end{lstlisting}

\paragraph{} You should now be able to type something similar to the following (assuming you named your file ``HelloWorld.java'':

\begin{lstlisting}[style=DOS]
    C:\> javac HelloWorld.java
    C:\> java HelloWorld
\end{lstlisting}

\paragraph{} This should of course be familiar to all of us, but a refresher never hurt anyone.

\subsection{Practise Programming}

\paragraph{} You can use any language that you like so use the opportunity to either practise a language you alread know or else to learn a new language. Remember, the more code that you write then the better you will be and the more your skills will improve. The best coders whom I have had the opportunity to work with were all good programmers purely because they put a lot of time and effort into getting better at what they do. 

\paragraph{} The task is to work through as many problems from [Project Euler](https://projecteuler.net/) as you like. Make sure to create a Git repository and add the code that you write for each problem to your repo. then push it to your remote account (Bitbucket or Github).

\paragraph{} I often use established problem lists as a way to learn a new language or to practise one I already know. There are many other similar sites that provide problem sets so if you can't think of a program to write then these places are a good starting point:

\begin{description}
\item[Project Euler]\url{https://projecteuler.net/}
\item[Stack Exchange Code Golf]\url{http://codegolf.stackexchange.com/}
\item[Code kata]\url{http://codekata.com/}
\item[Reddit Daily Programmer]\url{https://www.reddit.com/r/dailyprogrammer}
\item[Programming Praxis]\url{http://programmingpraxis.com/}
\item[Rosetta Code]\url{http://rosettacode.org/wiki/Main_Page}
\item[International Collegiate Programming Contest Problems Index]\url{http://acm. hit.edu.cn/judge/ProblemIndex.php}
\item[Algorithmist]\url{http://www.algorithmist.com/index.php/Main_Page}
\end{description}


\subsection{Background Reading}

\paragraph{} Use your browser to visit \url{http://link.springer.com}. If you are on the university network you shouldn't have to log in, but otherwise your university credentials will work with the Shibboleth log in option on the site. Use the search option with the following query: ``Algorithms and Data Structures'', you might want to narrow the result by selecting the ``Books'' option in the menu on the lefthand side.

\paragraph{} You can, and should, download PDF copies of the following books which will supplement your reading throughout the module.

\begin{itemize}
\item Algorithms and Data structures: The Basic Toolbox\\
\includegraphics[width=.8\textwidth]{images/mehlhorn}

\item Data Structures and Algorithms with Python\\
\includegraphics[width=.8\textwidth]{images/lee}

\item A Concise and Practical Introduction to Programming Algorithms in Java\\
\includegraphics[width=.8\textwidth]{images/nielsen}

\end{itemize}

\paragraph{} Pointers to other texts, chapters, and papers will be suggested throughout the module, but these should be enough to get started and provide a solid underpinning for your learning.

\subsection{\emph{CHALLENGE:} Use Git}
\paragraph{} Git is a source control system that enables you to keep track of your source code, its history and any changes you make. Git can be used to track any file but is most efficient and best suited when used only with textual files. Because Git is a \emph{distributed} source control system it works very well to enable groups of people to work on the same source code as well as supporting experimenting with your code, trying out lots of different ideas in separate \emph{branches} (which are a bit like a copy of your code but with tools to help manage that copy and support re-integrating it with your main source tree if you want to), and being able to roll back to an earlier version if you decide you have take a wrong turn.

\paragraph{} I'm not going to give detailed instrutions for setting up Git on the lab machines, that's why this is a \emph{challenge}.

\begin{framed}
\textbf{IMPORTANT} Git will be an option for submission of the coursework hand-in so if you intend to use this option then you should get familiar with it as soon as possible. A good place to start is by dipping into the Git SCM book\footnote{\url{http://git-scm.com/book/en/v2/Getting-Started-About-Version-Control}}.
\end{framed}

\paragraph{} There are also numerous interactive tutorials and resources to help you get started with Git:
\begin{enumerate}
\item Github's Learn Git 15 minute tutorial: \url{https://try.github.io/levels/1/challenges/1}
\item Learn Git Branching \url{http://pcottle.github.io/learnGitBranching/}
\item Git Immersion \url{http://gitimmersion.com/lab_01.html}
\end{enumerate}

\paragraph{} You should also create either a Github account\footnote{\url{https://github.com/}} or a Bitbucket account\footnote{\url{https://bitbucket.org}} (or both if you like) then create a repository within your new account called `set09117'. You will push all of your code throughout the module into this repository and at hand in time I will pull a copy for marking. The advantage of this appraoch is that at any point, if you need help with your code, then we have a copy that I can see remotely. However this only works if you keep adding your code to your repository. That means whenever you make changes you need to (1) add them, (2) commit them with an explanatory message,  and (3) push the changes from your local repository to the shared one on Github or Bitbucket.



%\backmatter

\bibliographystyle{plain}

\bibliography{workbook}

\end{document}

%\begin{framed}
%HELLO
%\end{framed}


